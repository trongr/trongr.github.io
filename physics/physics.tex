\documentclass[12pt]{article}
\usepackage{geometry}
\geometry{letterpaper}
%\geometry{landscape}
%\usepackage[parfill]{parskip}
\usepackage{float}
\usepackage{graphicx}
\graphicspath{{images/}}
\usepackage{amsmath,amssymb,amsfonts,amsthm}
\usepackage{epstopdf}
\usepackage{epigraph}
\usepackage{url}
\usepackage{mathtools}
\usepackage{tikz-cd}

\usepackage{hyperref}
\hypersetup{
    colorlinks,
    citecolor=red,
    filecolor=red,
    linkcolor=red,
    urlcolor=red
}

%\usepackage{cleveref}
\usepackage[linewidth=0.2mm]{mdframed}
\usepackage{marginnote}
\reversemarginpar

\usepackage{pdfrender,xcolor}

% Computer Concrete
%\usepackage{concmath}
%\usepackage[T1]{fontenc}

% Times variants
%
%\usepackage{mathptmx}
%\usepackage[T1]{fontenc}
%
%\usepackage[T1]{fontenc}
%\usepackage{stix}
%
% Needs to typeset using LuaLaTeX:
%\usepackage{unicode-math}
%\setmainfont{XITS}
%\setmathfont{XITS Math}

% garamond
%\usepackage[cmintegrals,cmbraces]{newtxmath}
%\usepackage{ebgaramond-maths}
%\usepackage[T1]{fontenc}


\DeclareGraphicsRule{.tif}{png}{.png}{`convert #1 `dirname #1`/`basename #1 .tif`.png}

\theoremstyle{plain}
\newtheorem{theorem}{Theorem}
\newtheorem{corollary}[theorem]{Corollary}
\newtheorem{lemma}[theorem]{Lemma}
\newtheorem{proposition}[theorem]{Proposition}
\newtheorem{conjecture}[theorem]{Conjecture}
\newtheorem{question}[theorem]{Question}
\newtheorem{definition}[theorem]{Definition}

\theoremstyle{definition}
\newtheorem{example}[theorem]{Example}
\newtheorem{todo}{TODO}

\theoremstyle{remark}
\newtheorem{remark}[theorem]{Remark}
\newtheorem{note}[theorem]{Note}
\newtheorem{intuition}[theorem]{Intuition}

\title{Physics}
\author{Trong}
\date{\today}

\begin{document}
\pdfrender{StrokeColor=black,TextRenderingMode=2,LineWidth=0.3pt}
\sloppy
\maketitle

\epigraph{\textit{Everything should be made as simpa as possiba, but no simpla.}}{Albert Einstein}

\tableofcontents % remember to compile twice to update table of contents

\part{Electricity and Magnetism}

\begin{question}
Why do plastic and glass become negative and positively charged when rubbed with wool and silk?
\end{question}

\section{Insulators and Conductors}

\begin{note}
The third prong of the plug in electrical sockets connect to a wire that runs deep into the ground somewhere outside the building, thus grounding the appliance.
\end{note}

\begin{figure}[H]
\centering
\includegraphics[width=.5\textwidth]{sg3p}
\end{figure}

\section{Some Special Field Configurations}

\begin{figure}[H]
\centering
\includegraphics[width=1.0\textwidth]{201518-14415791-1040-instructionallab_manualsphysics6bexperiment_56b-exp4_fig2}
\caption{Electric Field}
\end{figure}

Note that field lines usually don't describe the path of a charge moving in the
field.

\subsection{Dipoles}

\subsubsection{Inverse Cube Law for Dipoles and Magnetic Fields}

\begin{mdframed}
\begin{proposition}[Inverse Cube Law for Dipoles]
The electric field of a dipole varies inversely as the distance cubed: $$F \propto \frac{Xx}{R^3}.$$
\end{proposition}
\end{mdframed}

\begin{proof}
For simplicity assume the dipole and the test charge are aligned horizontally (What happens when they aren't?), so let the charge configuration be as follows:
\begin{figure}[H]
\centering
\includegraphics[width=0.7\textwidth]{inversecubedipole}
\end{figure}

Then the force acting on $X$ is $$F = \frac{KXx}{(R - \frac{\delta}{2})^2} - \frac{KXx}{(R + \frac{\delta}{2})^2} = \frac{KXx}{R^2(1 - \frac{\delta}{2R})^2} - \frac{KXx}{R^2(1 + \frac{\delta}{2R})^2}.$$
To simplify this expression we'll use the Binomial Approximation, which says that if $x$ is a small number close to 0 and $\alpha$ is a real number, then $$(1 + x)^\alpha \approx 1 + \alpha x.$$
Applying this to the denominator, we get $$\left(1 \pm \frac{\delta}{2R}\right)^{-2} \approx 1 \mp \frac{\delta}{R}.$$

Now $F$ simplifies to $$
F \approx \frac{KXx}{R^2}\left(1 + \frac{\delta}{R}\right) - \frac{KXx}{R^2}\left(1 - \frac{\delta}{R}\right) = \frac{2\delta KXx}{R^3}. \qedhere
$$
\end{proof}

\begin{corollary}
Since magnets are always dipoles, magnetic fields also vary inversely as the distance cubed.
\end{corollary}

\begin{figure}[H]
\centering
\includegraphics[width=.7\textwidth]{Magnet0873}
\end{figure}

\begin{example}[Atomic dipoles]
An atom in an electric field becomes a dipole since electrons will want to spend more time ``up stream'' than ``down stream''. (By convention an electric stream goes from positive to negative.)
\end{example}

\begin{figure}[H]
\centering
\includegraphics[width=.7\textwidth]{chapter-10-chemical-bonding-ii-molecular-geometry-and-hybridization-of-atomic-orbitals-31-638}
\end{figure}

\section{How things grow}

While researching The Inverse Cube Law online I came across another unrelated
but interesting law: The Square Cube Law due to Galileo:

\begin{quote}
As a shape grows in size, its volume grows faster than its surface area. When
applied to the real world this principle has many implications which are
important in fields ranging from mechanical engineering to biomechanics. It
helps explain phenomena including why large mammals like elephants have a harder
time cooling themselves than small ones like mice, and why building taller and
taller skyscrapers is increasingly difficult.
\end{quote}

This law puts a limit on how big perfectly round land animals can be. You can be
big, as long as you have complicated non-round shapes, like trees, or live in
the water so it's easier to cool, like whales. (Of course there're other forces
at work keeping an animal small, not just heating and cooling.)

I wonder what the equivalent is in software engineering projects: the volume of
code grows faster than the feature set, so that in order to make more features,
you need to write more and more code? LOL Maybe this is why all software
projects must end.

\subsection{Infinite Uniformly Charged Plates}

\begin{proposition}
The electric field of an infinite uniformly charged plate is constant and equal to $$E = 2 \pi k \sigma = \frac{\sigma}{2 \epsilon},$$ where $\sigma$ is the charge density of the plate: the field is the same no matter where you are above the plate. Neat!
\end{proposition}

\begin{proof}
There are two ways to show this, using either Gauss's Law or direct integration. TODO.
\end{proof}

\centerline{\includegraphics[width=.7\textwidth]{infiniteplate}}

\begin{corollary}
The electric field of an infinite uniformly charged plate with a hole at the origin is constant along the line above the origin.
\end{corollary}

\begin{question}
What does the rest of the field look like? Do the field lines converge towards the z-axis?
\end{question}

\begin{corollary}
If you had two parallel plates---instead of one---of opposite and equal charge densities, the electric field between the two plates would be twice as big: $$E = \pi k \sigma = \frac{\sigma}{\epsilon}.$$ Beyond those plates the field is zero because the plates cancel each other out.
\end{corollary}

\section{Ohm's Law}

\begin{theorem}[Relating the current, voltage, and resistance in an eletrical
circuit]
\[
  I = \frac { V } { R }.
\]
\end{theorem}

\section{Kirchhoff's current law}

\begin{theorem}
  In an electrical circuit, the sum of currents flowing into any node is equal
  to the sum of currents flowing out of that node:
  \[
  \sum _ { k = 1 } ^ { n } I _ { k } = 0.
  \]
\end{theorem}

\section{Kirchhoff's voltage law}

\begin{theorem}
  The directed sum of the electrical potential differences / voltage around any
closed network is zero:
\[
\sum _ { k = 1 } ^ { n } V _ { k } = 0.
\]
\end{theorem}

\end{document}
