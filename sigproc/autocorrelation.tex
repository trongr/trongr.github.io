\documentclass[12pt]{article}
\usepackage{geometry}                % See geometry.pdf to learn the layout options. There are lots.
\geometry{letterpaper}                   % ... or a4paper or a5paper or ... 
%\geometry{landscape}                % Activate for for rotated page geometry
%\usepackage[parfill]{parskip}    % Activate to begin paragraphs with an empty line rather than an indent
\usepackage{float}
\usepackage{graphicx}
\graphicspath{{images/}}
\usepackage{amsmath,amssymb,amsfonts,amsthm} 
\usepackage{epstopdf}
\usepackage{cleveref}
\usepackage{epigraph}

% Computer Concrete
%\usepackage{concmath}
%\usepackage[T1]{fontenc}

% Times variants
%
\usepackage{mathptmx}
\usepackage[T1]{fontenc}
%
%\usepackage[T1]{fontenc}
%\usepackage{stix}
%
% Needs to typeset using LuaLaTeX:
%\usepackage{unicode-math}
%\setmainfont{XITS}
%\setmathfont{XITS Math}

% garamond
%\usepackage[cmintegrals,cmbraces]{newtxmath}
%\usepackage{ebgaramond-maths}
%\usepackage[T1]{fontenc}

\DeclareGraphicsRule{.tif}{png}{.png}{`convert #1 `dirname #1`/`basename #1 .tif`.png}

\theoremstyle{plain}
\newtheorem{theorem}{Theorem}
\newtheorem{corollary}[theorem]{Corollary}
\newtheorem{lemma}[theorem]{Lemma}
\newtheorem{proposition}[theorem]{Proposition}
\newtheorem{conjecture}[theorem]{Conjecture}
\newtheorem{question}[theorem]{Question}

\theoremstyle{definition}
\newtheorem{definition}[theorem]{Definition}
\newtheorem{example}[theorem]{Example}
\newtheorem*{keywords}{Keywords}
\newtheorem*{sources}{Sources}
\newtheorem*{todo}{TODO}

\theoremstyle{remark}
\newtheorem{remark}[theorem]{Remark}
\newtheorem{note}[theorem]{Note}

\title{Autocorrelation}
\author{Nhan Trong}
\date{\today}                                           % Activate to display a given date or no date

\begin{document}
\maketitle

\epigraph{\textit{If you want to find the secrets of the universe, think in terms of energy, frequency and vibration.}}{Nikola Tesla}

\begin{keywords}
Standard, cuberoot and enhanced autocorrelation.
\end{keywords}

\begin{question}
Can you use autocorrelation to detect speech events?
\end{question}

It appears you can. As an example, Audacity has an implementation called Standard Autocorrelation (the other two are Cuberoot and Enhanced), which we can apply to non-vocal and vocal segments of an audio:

\begin{figure}[H]
\centering
\includegraphics[width=1.0\textwidth]{autocorrelation1}
\end{figure}

\begin{figure}[H]
\centering
\includegraphics[width=1.0\textwidth]{autocorrelation2}
\end{figure}

The non-vocal graph drops off smoothly over the next 5 milliseconds or so, whereas the vocal graph is more bumpy, presumably because the human voice is more textured and contains more repeating patterns than most other sounds, including even music.

\begin{todo}
Figure out how autocorrelation works.
\end{todo}

\begin{question}
Might we also be able to identify different voices?
\end{question}

\end{document}
