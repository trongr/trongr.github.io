% Converting a PDF into JPGs:
%
% gs -dFirstPage=26 -dLastPage=33 -dBATCH -dNOPAUSE -sDEVICE=jpeg -r666 -dUseCropBox -o images/OUT-%04d.jpg IN.pdf

% Alternatively, output at a lower resolution, and use convert to blur, for a
% more book-like effect:
%
% gs -dFirstPage=1 -dLastPage=10 -dBATCH -dNOPAUSE -sDEVICE=jpeg -r150 -dUseCropBox -o images/RLII-%04d.jpg ReinforcementLearningII.pdf
% mogrify -blur 0x0.5 images/RLII-*.jpg

% \documentclass[12pt]{article}
% \documentclass[14pt]{extarticle}
% \documentclass[17pt]{extarticle}
\documentclass[20pt]{extarticle}
\usepackage[margin=1cm,paperwidth=7in,paperheight=7in]{geometry}
% \usepackage[margin=1cm,paperwidth=8in,paperheight=10in]{geometry}
% \usepackage[a4paper,margin=1cm,paperwidth=8.5in,paperheight=11in]{geometry}

% Font sizes for reference
% \tiny
% \scriptsize
% \footnotesize
% \small
% \normalsize
% \large
% \Large
% \LARGE
% \huge
% \Huge

\usepackage{float}
\usepackage{graphicx}
\graphicspath{{images/}}

\usepackage{listings}
\lstset{basicstyle=\ttfamily, aboveskip=\bigskipamount, belowskip=\bigskipamount}


\usepackage{amsmath,amssymb,amsfonts,amsthm}
\usepackage{epstopdf}
\usepackage{epigraph}
\usepackage{url}
\usepackage{mathtools}

\usepackage{hyperref}
\hypersetup{
    colorlinks,
    citecolor=red,
    filecolor=red,
    linkcolor=red,
    urlcolor=red
}

\usepackage{pdfrender,xcolor}

\usepackage{mathtools} % Lets you use under and over square brackets

\usepackage{lmodern}
\usepackage[T1]{fontenc}

% Computer Concrete
%\usepackage{concmath}
%\usepackage[T1]{fontenc}

% Times variants

% \usepackage{mathptmx}
% \usepackage[T1]{fontenc}

% \usepackage[T1]{fontenc}
% \usepackage{stix}

% Needs to typeset using LuaLaTeX:

% \usepackage{unicode-math}
% \setmainfont{XITS}
% \setmathfont{XITS Math}

% garamond

% \usepackage[cmintegrals,cmbraces]{newtxmath}
% \usepackage{ebgaramond-maths}
% \usepackage[T1]{fontenc}

\DeclareGraphicsRule{.tif}{png}{.png}{`convert #1 `dirname #1`/`basename #1 .tif`.png}

% Environments

% Redefine enumerate environment to have less spacing between items
\let\oldBeginEnumerate=\enumerate
\let\oldEndEnumerate=\endenumerate
\renewenvironment{enumerate}{
\oldBeginEnumerate
  \setlength{\itemsep}{0pt}
  \setlength{\parskip}{0pt}
  \setlength{\parsep}{0pt}
}{\oldEndEnumerate}

% Italics body
\theoremstyle{plain}
\newtheorem{theorem}{Theorem}
\newtheorem{corollary}[theorem]{Corollary}
\newtheorem{lemma}[theorem]{Lemma}
\newtheorem{proposition}[theorem]{Proposition}
\newtheorem{conjecture}[theorem]{Conjecture}
\newtheorem{question}[theorem]{Question}
\newtheorem{problem}[theorem]{Problem}
\newtheorem{exercise}[theorem]{Exercise}
\newtheorem{definition}[theorem]{Definition}

% Regular body
\theoremstyle{definition}
\newtheorem{example}[theorem]{Example}
\newtheorem{keywords}{Keywords}
\newtheorem{reference}{Reference}

\theoremstyle{remark}
\newtheorem{remark}[theorem]{Remark}
\newtheorem{note}[theorem]{Note}

\newenvironment{solution}{\begin{proof}[Solution]}{\end{proof}}

\renewcommand{\qedsymbol}{$\blacksquare$}

% Thicker fraction divider
\renewcommand\frac[2]{\genfrac{}{}{1pt}{}{#1}{#2}}

% Operators

\DeclareMathOperator{\argmax}{argmax}

% Bold math

\newcommand{\bE}{\mathbf E}
\newcommand{\bN}{\mathbf N}
\newcommand{\bQ}{\mathbf Q}
\newcommand{\bR}{\mathbf R}

% Blackboard (hollow) math

\newcommand{\hE}{\mathbb E}
\newcommand{\hP}{\mathbb P}

% Greek and misc symbols

\newcommand{\0}{\varnothing}

\renewcommand{\a}{\alpha}
\renewcommand{\b}{\beta}
\newcommand{\e}{\varepsilon}
\newcommand{\f}{\varphi}
\newcommand{\g}{\gamma}
\newcommand{\m}{\mu}
\newcommand{\s}{\sigma}
\renewcommand{\th}{\theta}

\newcommand{\D}{\Delta}

% Delimiters

%\newcommand{\defeq}{\coloneqq}
\newcommand*{\defeq}{\mathrel{\vcenter{\baselineskip0.5ex \lineskiplimit0pt
                     \hbox{\scriptsize.}\hbox{\scriptsize.}}}
                     =}
\newcommand{\<}{\langle}
\renewcommand{\>}{\rangle}

\title{Number Theory}
\author{Trong}
\date{Sat Dec 1, 2018---\today}

\begin{document}
\pdfrender{StrokeColor=black,TextRenderingMode=2,LineWidth=0.5pt}
\sloppy
\maketitle
\thispagestyle{empty}
\pagestyle{empty}

% \usepackage{hyperref} for hyper links in table of contents. Remember to
% compile twice to update table of contents
\tableofcontents
\break

\section{Congruences}

\begin{theorem}[Divisor Sum]
  For any natural number $ n, $
\[
\sum _ { d | n } \f ( d ) = n,
\]
where \( \f(d) \) is the Euler Totient function.
\end{theorem}

\begin{proof}
Consider the set \( A(d) = \{ k: (k, n) = d \}. \) For each \( k, \) define \( l \) s.t. \( k = dl. \) Then it's easy to see that \( (l, \frac{n}{d}) = 1. \) In fact, there is a one-to-one correspondence between \( k \) and \( l, \) so that \( | A(d) | = | \{ k \} | = | \{ l \} |. \) Now the \( l \)'s are numbers less than \( \frac{n}{d} \) and coprime with it, so \( | A(d) | = \f(\frac{n}{d}). \)

Next, note that the sets \( A(d) \) for distinct \( d|n \) are disjoint and their union is \( {1, \ldots, n}. \) Therefore
\[
n = \sum_{d|n} |A(d)| = \sum_{d|n} \f\left(\frac{n}{d}\right).
\]

Finally
\[
n = \sum_{d|n} \f\left(\frac{n}{d}\right) = \sum_{d|n} \f(d),
\]
since the divisors \( \frac{n}{d} \) in the first sum are the same as the divisors \( d \) in the second sum.
\end{proof}

\begin{proposition}[NZM Ex. 2.1.15.]
Find integers $ a_1, \ldots, a_5 $ s.t. every integer $ x $ satisfies at least one of the congruences \begin{align*}
  x &\equiv a_1 \bmod 2 \\
  x &\equiv a_2 \bmod 3 \\
  x &\equiv a_3 \bmod 4 \\
  x &\equiv a_4 \bmod 6 \\
  x &\equiv a_5 \bmod 12. \tag{$ * $}
\end{align*}
\end{proposition}

\begin{solution}
Consider the remainder classes mod 3: \begin{align*}
  & 3 n \\
  & 3 n + 1 \\
  & 3 n + 2.
\end{align*}
Substitute $ 2k $ and $ 2k + 1 $ for $ n, $ and take their remainders mod 2, 3, and 6: \begin{align*}
3 \cdot 2 k &\equiv 0 \bmod 2 \\
3 (2k + 1) = 6k + 3 &\equiv 0 \bmod 3 \\
3 \cdot 2k + 1 = 6k + 1 &\equiv 1 \bmod 6 \\
3 (2k + 1) + 1 = 6k + 4 &\equiv 0 \bmod 2 \\
3 \cdot 2k + 2 &\equiv 0 \bmod 2 \\
3 (2k + 1) + 2 = 6k + 5 &\equiv 5 \bmod 6.
\end{align*}
We've now covered every integer with mods 2, 3, and 6; if we can somehow write integers $ 5 \bmod 6 $ as either $ a_3 \bmod 4 $ or $ a_5 \bmod 12, $ then we will have expressed every integer in the form $(*).$ Let's do that: \begin{align*}
6 \cdot 2k + 5 = 12k + 5 = 4(3k + 1) + 1 &\equiv 1 \bmod 4 \\
6(2k + 1) + 5 = 12k + 11 &\equiv 11 \bmod 12.
\end{align*}

Therefore every integer $ x $ satisfies at least one of
\begin{align*}
  x &\equiv 0 \bmod 2 \\
  x &\equiv 0 \bmod 3 \\
  x &\equiv 1 \bmod 4 \\
  x &\equiv 1 \bmod 6 \\
  x &\equiv 11 \bmod 12. \qedhere
\end{align*}
\end{solution}

\begin{theorem}[NZM 2.9]
If $ (a, m) = 1, $ then there is an $ x $ s.t. $ ax \equiv 1 \bmod m. $ Any two such $ x $ are congruent mod $ m. $ If $ (a, m) > 1,$ then there is no such $ x. $
\end{theorem}

In other words, if $ a $ and $ m $ are relatively prime, then $ a $ has an inverse mod $ m. $

\begin{theorem}[Wilson's Theorem]
If $ p $ is prime, then $ p - 1 \equiv -1 \bmod p. $
\end{theorem}

\begin{proposition}[NZM Ex. 2.1.34. Wilson's Theorem revisited]
An integer $ p > 1 $ is prime iff $ p | (p - 1)! + 1. $
\end{proposition}

\begin{proof}
Suppose $ p $ is prime. By Wilson's Theorem, \[
p - 1 \equiv -1 \bmod p. \tag{WT}
\]
We want to show that \begin{align*}
(p - 1)! &\equiv -1 \bmod p \\
(p - 1) \underbracket{(p - 2)(p - 3) \cdots 1}_{G} &\equiv -1. \tag{WT2}
\end{align*}
Since $ p $ is prime, by NZM 2.9, every factor in $ G $ has an inverse mod $ p $ in $ G, $ so they cancel each other out. Therefore we can go back and forth between $ WT $ and $ WT2. $

Conversely, suppose that $ p | (p - 1)! + 1 $ and $ p = aq $ is composite. Then \[
(p - 1)! + 1 = aqk
\]
for some $ k. $ Now note that $ a $ divides the RHS, and also the first term on the LHS, therefore it must divide the $ 1 $ on the LHS, which is impossible since $ a \neq 1. $
\end{proof}

\end{document}
