\documentclass[17pt]{extarticle}
\usepackage{geometry}
\geometry{letterpaper}
\usepackage{float}
\usepackage{graphicx}
\graphicspath{{images/}}
\usepackage{amsmath,amssymb,amsfonts,amsthm}
\usepackage{epstopdf}
\usepackage{epigraph}
\usepackage{url}
\usepackage{mathtools}

\usepackage{hyperref}
\hypersetup{
    colorlinks,
    citecolor=red,
    filecolor=red,
    linkcolor=red,
    urlcolor=red
}

\usepackage{pdfrender,xcolor}

\usepackage{lmodern}
\usepackage[T1]{fontenc}

% Computer Concrete
% \usepackage{concmath}
% \usepackage[T1]{fontenc}

% Times variants
%
% \usepackage{mathptmx}
% \usepackage[T1]{fontenc}
%
% \usepackage[T1]{fontenc}
% \usepackage{stix}
%
% Needs to typeset using LuaLaTeX:
% \usepackage{unicode-math}
% \setmainfont{XITS}
% \setmathfont{XITS Math}

% garamond
% \usepackage[cmintegrals,cmbraces]{newtxmath}
% \usepackage{ebgaramond-maths}
% \usepackage[T1]{fontenc}

\DeclareGraphicsRule{.tif}{png}{.png}{`convert #1 `dirname #1`/`basename #1 .tif`.png}

% Environments

% Redefine enumerate environment to have less spacing between items
\let\oldBeginEnumerate=\enumerate
\let\oldEndEnumerate=\endenumerate
\renewenvironment{enumerate}{
\oldBeginEnumerate
  \setlength{\itemsep}{0pt}
  \setlength{\parskip}{0pt}
  \setlength{\parsep}{0pt}
}{\oldEndEnumerate}

% Italics body
\theoremstyle{plain}
\newtheorem{theorem}{Theorem}
\newtheorem{corollary}[theorem]{Corollary}
\newtheorem{lemma}[theorem]{Lemma}
\newtheorem{proposition}[theorem]{Proposition}
\newtheorem{conjecture}[theorem]{Conjecture}
\newtheorem{question}[theorem]{Question}
\newtheorem{problem}[theorem]{Problem}
\newtheorem{exercise}[theorem]{Exercise}

% Regular body
\theoremstyle{definition}
\newtheorem{definition}[theorem]{Definition}
\newtheorem{example}[theorem]{Example}
\newtheorem{keywords}{Keywords}
\newtheorem{reference}{Reference}

\theoremstyle{remark}
\newtheorem{remark}[theorem]{Remark}
\newtheorem{note}[theorem]{Note}

\newenvironment{solution}{\begin{proof}[Solution]}{\end{proof}}

% Operators

\DeclareMathOperator{\argmax}{argmax}

% Bold math

\newcommand{\bN}{\mathbf N}
\newcommand{\bQ}{\mathbf Q}
\newcommand{\bR}{\mathbf R}

% Blackboard (hollow) math

\newcommand{\hE}{\mathbb E}
\newcommand{\hP}{\mathbb P}

% Greek and misc symbols

\newcommand{\0}{\varnothing}
\newcommand{\e}{\varepsilon}
\newcommand{\f}{\varphi}

% Delimiters

%\newcommand{\defeq}{\coloneqq}
\newcommand*{\defeq}{\mathrel{\vcenter{\baselineskip0.5ex \lineskiplimit0pt
                     \hbox{\scriptsize.}\hbox{\scriptsize.}}}
                     =}
\newcommand{\<}{\langle}
\renewcommand{\>}{\rangle}

\title{Number Theory}
\author{Trong}
\date{Sat Dec 1, 2018---\today}

\begin{document}
\pdfrender{StrokeColor=black,TextRenderingMode=2,LineWidth=0.3pt}
\sloppy
\maketitle
\tableofcontents % remember to compile twice to update table of contents
\pagebreak

\section{Congruences}

\begin{theorem}[Divisor Sum]
  For any natural number $ n, $
\[
\sum _ { d | n } \f ( d ) = n,
\]
where \( \f(d) \) is the Euler Totient function.
\end{theorem}

\begin{proof}
Consider the set \( A(d) = \{ k: (k, n) = d \}. \) For each \( k, \) define \( l \) s.t. \( k = dl. \) Then it's easy to see that \( (l, \frac{n}{d}) = 1. \) In fact, there is a one-to-one correspondence between \( k \) and \( l, \) so that \( | A(d) | = | \{ k \} | = | \{ l \} |. \) Now the \( l \)'s are numbers less than \( \frac{n}{d} \) and coprime with it, so \( | A(d) | = \f(\frac{n}{d}). \)

Next, note that the sets \( A(d) \) for distinct \( d|n \) are disjoint and their union is \( {1, \ldots, n}. \) Therefore
\[
n = \sum_{d|n} |A(d)| = \sum_{d|n} \f\left(\frac{n}{d}\right).
\]

Finally
\[
n = \sum_{d|n} \f\left(\frac{n}{d}\right) = \sum_{d|n} \f(d),
\]
since the divisors \( \frac{n}{d} \) in the first sum are the same as the divisors \( d \) in the second sum.
\end{proof}

\begin{problem}
Find integers $ a_1, \ldots, a_5 $ s.t. every integer $ x $ satisfies at least one of the congruences \begin{align*}
  x &\equiv a_1 \bmod 2 \\
  x &\equiv a_2 \bmod 3 \\
  x &\equiv a_3 \bmod 4 \\
  x &\equiv a_4 \bmod 6 \\
  x &\equiv a_5 \bmod 12. \tag{$ * $}
\end{align*}
\end{problem}

\begin{solution}
Consider the remainder classes mod 3: \begin{align*}
  & 3 n \\
  & 3 n + 1 \\
  & 3 n + 2.
\end{align*}
Substitute $ 2k $ and $ 2k + 1 $ for $ n, $ and take their remainders mod 2, 3, and 6: \begin{align*}
3 \cdot 2 k &\equiv 0 \bmod 2 \\
3 (2k + 1) = 6k + 3 &\equiv 0 \bmod 3 \\
3 \cdot 2k + 1 = 6k + 1 &\equiv 1 \bmod 6 \\
3 (2k + 1) + 1 = 6k + 4 &\equiv 0 \bmod 2 \\
3 \cdot 2k + 2 &\equiv 0 \bmod 2 \\
3 (2k + 1) + 2 = 6k + 5 &\equiv 5 \bmod 6.
\end{align*}
We've now covered every integer with mods 2, 3, and 6; if we can somehow write integers $ 5 \bmod 6 $ as either $ a_3 \bmod 4 $ or $ a_5 \bmod 12, $ then we will have expressed every integer in the form $(*).$ Let's do that: \begin{align*}
6 \cdot 2k + 5 = 12k + 5 = 4(3k + 1) + 1 &\equiv 1 \bmod 4 \\
6(2k + 1) + 5 = 12k + 11 &\equiv 11 \bmod 12.
\end{align*}

Therefore every integer $ x $ satisfies at least one of
\begin{align*}
  x &\equiv 0 \bmod 2 \\
  x &\equiv 0 \bmod 3 \\
  x &\equiv 1 \bmod 4 \\
  x &\equiv 1 \bmod 6 \\
  x &\equiv 11 \bmod 12. \qedhere
\end{align*}
\end{solution}

\end{document}
