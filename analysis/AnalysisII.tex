% \documentclass[14pt]{extarticle}
\documentclass[17pt]{extarticle}
\usepackage{geometry}
\geometry{letterpaper}
\usepackage{float}
\usepackage{graphicx}
\graphicspath{{images/}}

\usepackage{amsmath,amssymb,amsfonts,amsthm}
\usepackage{epstopdf}
\usepackage{epigraph}
\usepackage{url}
\usepackage{mathtools}

\usepackage{pdfrender,xcolor}

\usepackage{hyperref}
\hypersetup{
    colorlinks,
    citecolor=red,
    filecolor=red,
    linkcolor=red,
    urlcolor=red
}

\usepackage{lmodern}
\usepackage[T1]{fontenc}

% Computer Concrete
% \usepackage{concmath}
% \usepackage[T1]{fontenc}

% Times variants
%
% \usepackage{mathptmx}
% \usepackage[T1]{fontenc}
%
% \usepackage[T1]{fontenc}
% \usepackage{stix}
%
% Needs to typeset using LuaLaTeX:
% \usepackage{unicode-math}
% \setmainfont{XITS}
% \setmathfont{XITS Math}

% garamond
% \usepackage[cmintegrals,cmbraces]{newtxmath}
% \usepackage{ebgaramond-maths}
% \usepackage[T1]{fontenc}

\DeclareGraphicsRule{.tif}{png}{.png}{`convert #1 `dirname #1`/`basename #1 .tif`.png}

\theoremstyle{plain}
\newtheorem{theorem}{Theorem}
\newtheorem{corollary}[theorem]{Corollary}
\newtheorem{lemma}[theorem]{Lemma}
\newtheorem{proposition}[theorem]{Proposition}
\newtheorem{conjecture}[theorem]{Conjecture}
\newtheorem{question}[theorem]{Question}

\theoremstyle{definition}
\newtheorem{definition}[theorem]{Definition}
\newtheorem{example}[theorem]{Example}
\newtheorem{keywords}{Keywords}
\newtheorem{reference}{Reference}

\theoremstyle{remark}
\newtheorem{remark}[theorem]{Remark}
\newtheorem{note}[theorem]{Note}

%\newcommand{\defeq}{\coloneqq}
\newcommand*{\defeq}{\mathrel{\vcenter{\baselineskip0.5ex \lineskiplimit0pt
                     \hbox{\scriptsize.}\hbox{\scriptsize.}}}
                     =}

\newcommand{\bN}{\mathbf N}
\newcommand{\bQ}{\mathbf Q}
\newcommand{\bR}{\mathbf R}

\title{Analysis II}
\author{Trong}
\date{December 1, 2018---\today}

\begin{document}
\pdfrender{StrokeColor=black,TextRenderingMode=2,LineWidth=0.3pt}
\sloppy
\maketitle

% \usepackage{hyperref} for hyper links in table of contents. Remember to
% compile twice to update table of contents
\tableofcontents

\section{Differentiation}

\begin{definition}[Directional derivative]
  Let \( A \subset \bR^m, f: A\longrightarrow \bR^n. \) Suppose $A$ contains a neighbourhood of \( a. \) Given \( u \in \bR^m \) with \( u \neq 0, \) define the directional derivative of \( f \) at \( a \) in the direction of \( u \) to be
  \[
  f'(a; u) = \lim_{t\to 0} \frac{f(a+u) - f(a)}{t}
  \]
if it exists.
\end{definition}

\begin{definition}[Derivative]
  Let \( A \subset \bR^m, f: A\longrightarrow \bR^n. \) Suppose $A$ contains a
  neighbourhood of a. We say that \( f \) is differentiable at \( a \) if there
  is an \( n\times m \) matrix $B$ s.t.
\[
\frac{f(a+h) - f(a) - B\cdot h}{|h|} \longrightarrow 0
\]
as \( h \longrightarrow 0. \) The matrix $B$ is unique and is denoted \( D f(a).
\) Sometimes people also call the derivative the gradient, and write \( D f(a) =
\nabla f(a). \)
\end{definition}

\begin{theorem}[Relating directional derivatives to the derivative of \( f \)]
  Let \( A \subset \bR^m, f: A\longrightarrow \bR^n. \) If f is differentiable at \( a, \) then all the directional derivatives of \( f \) at \( a \) exists, and
\[
f'(a; u) = D f(a) \cdot u.
\]
\end{theorem}

\begin{definition}[Partial derivative]
  Let \( A \subset \bR^m, f: A \longrightarrow \bR. \) Define the \( j \)-th partial derivative of \( f \) at \( a \) to be directional derivative of \( f \) at \( a \) with respect to the vector \( e_j, \) provided it exists; and we denote it by \( D_j f(a): \)
  \[
  D_j f(a) = \lim_{t\to 0} \frac{f(a + t e_j) - f(a)}{t}.
  \]
\end{definition}

IOW, partial derivatives are directional derivatives along coordinate axes. Note
that if we define \( \phi(t) = f(a_1,\ldots, a_{j-1}, t, a_{j+1},\ldots,a_m), \)
then
\[
D_j f(a) = \phi'(a_j).
\]

\begin{theorem}[Derivative of a real-valued function]
  Let \( A \subset \bR^m, f: A \longrightarrow \bR. \) If \( f \) is differentiable at \( a, \) then the derivative of \( f \) is the row matrix
\[
Df(a) = [D_1 f(a) \quad \cdots \quad D_m f(a)].
\]
\end{theorem}

\begin{theorem}
Let \( A \subset \bR^m, f: A \longrightarrow \bR^n. \) Suppose \( A \) contains a neighbourhood of \( a. \) Let \( f_i: A \longrightarrow R \) be the \( i \)-th component function of \( f, \) so that
\[
f(x) = \begin{bmatrix}
f_1 (x) \\
\vdots \\
f_n (x)
\end{bmatrix}.
\]
\begin{itemize}
\item Then \( f \) is differentiable at \( a \) iff each component \( f_i \) is differentiable at \( a. \)

\item If \( f \) is differentiable at \( a, \) then its derivative is the \( n \times m \) matrix whose \( i \)-th row is the derivative of \( f_i,  \) i.e.
\[
D f(a) = \begin{bmatrix}
D f_1 (a) \\
\vdots \\
D f_n (a)
\end{bmatrix} = \begin{bmatrix}
D_1 f_1 (a) & \cdots & D_m f_1 (a) \\
\vdots & & \vdots \\
D_1 f_n (a) & \cdots & D_m f_n (a)
\end{bmatrix}.
\]
  This matrix is called the Jacobian matrix of \( f. \)
\end{itemize}
\end{theorem}

Roughly: Differentiability of \( f: \bR^m \longrightarrow \bR^n \) is equivalent
to differentiability of each component, because the components are independent
of each other as far as taking limits is concerned. Note that this doesn't imply
that the partial derivatives of the components must be continuous, only that
they exist.

\section{Continuously differentiable functions}

\begin{theorem}[Mean value theorem]
If $ \phi: [a, b] \longrightarrow \bR $ is continuous at each point of the closed interval $ [a, b], $ and differentiable at each point of the interval $ (a, b), $ then there exists a point $ c $ of $ (a, b) $ s.t. $$
  \phi(b) - \phi(a) = \phi'(c) (b - a).
$$
\end{theorem}

\begin{theorem}[Continuously differentiable functions]
  Let $ A $ be open in $ \bR^m. $ Suppose that the partial derivatives $ D_j f_i(x) $ of the component functions of $ f $ exist at each point $ x \in A $ and are continuous on $ A. $ Then $ f $ is differentiable at each point of $ A. $
\end{theorem}

This theorem guarantees differentiability of $ f $ if its partial derivatives exist and are continuous. Such a function is called continuously differentiable, or $ C^1 $ on $ A. $

\begin{theorem}
  Let $ A $ be open in $ \bR^m, f: A \longrightarrow \bR $ be a function of class $ C^2. $ Then for each $ a \in A, $ the mixed second order partial derivatives are equal: $$
    D_k D_j f(a) = D_j D_k f(a).
  $$
\end{theorem}

\end{document}
