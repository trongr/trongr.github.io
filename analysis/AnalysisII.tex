% Converting a PDF into JPGs:
%
% gs -dFirstPage=26 -dLastPage=33 -dBATCH -dNOPAUSE -sDEVICE=jpeg -r666 -dUseCropBox -o images/OUT-%04d.jpg IN.pdf

% Alternatively, output at a lower resolution, and use convert to blur, for a
% more book-like effect:
%
% gs -dFirstPage=1 -dLastPage=10 -dBATCH -dNOPAUSE -sDEVICE=jpeg -r150 -dUseCropBox -o images/RLII-%04d.jpg ReinforcementLearningII.pdf
% mogrify -blur 0x0.5 images/RLII-*.jpg

\documentclass[12pt]{article}
% \documentclass[14pt]{extarticle}
% \documentclass[17pt]{extarticle}
% \documentclass[20pt]{extarticle}
% \usepackage[margin=0.5cm,paperwidth=4in,paperheight=4in]{geometry}
% \usepackage[margin=1cm,paperwidth=7in,paperheight=7in]{geometry}
% \usepackage[margin=1cm,paperwidth=8in,paperheight=10in]{geometry}
% \usepackage[a4paper,margin=1cm,paperwidth=8.5in,paperheight=11in]{geometry}

% Font sizes for reference
% \tiny
% \scriptsize
% \footnotesize
% \small
% \normalsize
% \large
% \Large
% \LARGE
% \huge
% \Huge

% Languages: English, Vietnamese, Chinese, Japanese, Korean.

% \usepackage[english,vietnamese]{babel}
% \usepackage[utf8]{inputenc}
% \usepackage[T1,T2A]{fontenc}
% \usepackage{CJKutf8}
% % Options for fonts: Chinese: gbsn, gkai; Japanese: min, goth, maru
% \begin{CJK}{UTF8}{gbsn}天下人都\end{CJK}

\usepackage{float}
\usepackage{graphicx}
\graphicspath{{images/}}

\usepackage{listings}
\lstset{basicstyle=\ttfamily, aboveskip=\bigskipamount, belowskip=\bigskipamount}

\usepackage{amsmath,amssymb,amsfonts,amsthm}
\usepackage{epstopdf}
\usepackage{epigraph}
\usepackage{url}
\usepackage{mathtools}

\usepackage{hyperref}
\hypersetup{
    colorlinks,
    citecolor=red,
    filecolor=red,
    linkcolor=red,
    urlcolor=red
}

\usepackage{pdfrender,xcolor}

\usepackage{lmodern}
\usepackage[T1]{fontenc}

% Computer Concrete
%\usepackage{concmath}
%\usepackage[T1]{fontenc}

% Times variants

% \usepackage{mathptmx}
% \usepackage[T1]{fontenc}

% \usepackage[T1]{fontenc}
% \usepackage{stix}

% Needs to typeset using LuaLaTeX:

% \usepackage{unicode-math}
% \setmainfont{XITS}
% \setmathfont{XITS Math}

% garamond

% \usepackage[cmintegrals,cmbraces]{newtxmath}
% \usepackage{ebgaramond-maths}
% \usepackage[T1]{fontenc}

\DeclareGraphicsRule{.tif}{png}{.png}{`convert #1 `dirname #1`/`basename #1 .tif`.png}

% Environments

% Redefine enumerate environment to have less spacing between items
\let\oldBeginEnumerate=\enumerate
\let\oldEndEnumerate=\endenumerate
\renewenvironment{enumerate}{
\oldBeginEnumerate
  \setlength{\itemsep}{0pt}
  \setlength{\parskip}{0pt}
  \setlength{\parsep}{0pt}
}{\oldEndEnumerate}

% The same for itemize environment.

\let\oldBeginItemize=\itemize
\let\oldEndItemize=\enditemize
\renewenvironment{itemize}{
\oldBeginItemize
  \setlength{\itemsep}{0pt}
  \setlength{\parskip}{0pt}
  \setlength{\parsep}{0pt}
}{\oldEndItemize}

% Italics body
\theoremstyle{plain}
\newtheorem{theorem}{Theorem}
\newtheorem{corollary}[theorem]{Corollary}
\newtheorem{lemma}[theorem]{Lemma}
\newtheorem{proposition}[theorem]{Proposition}
\newtheorem{conjecture}[theorem]{Conjecture}
\newtheorem{question}[theorem]{Question}
\newtheorem{problem}[theorem]{Problem}
\newtheorem{exercise}[theorem]{Exercise}
\newtheorem{definition}[theorem]{Definition}

% Regular body
\theoremstyle{definition}
\newtheorem{example}[theorem]{Example}
\newtheorem{keywords}{Keywords}
\newtheorem{reference}{Reference}

\theoremstyle{remark}
\newtheorem{remark}[theorem]{Remark}
\newtheorem{note}[theorem]{Note}

\newenvironment{solution}{\begin{proof}[Solution]}{\end{proof}}

% Black square for QED symbol
\renewcommand{\qedsymbol}{$\blacksquare$}

% Thicker fraction divider
\renewcommand\frac[2]{\genfrac{}{}{0.6pt}{}{#1}{#2}}

% Operators

\DeclareMathOperator{\argmax}{argmax}

% Bold math

\newcommand{\bE}{\mathbf E}
\newcommand{\bN}{\mathbf N}
\newcommand{\bQ}{\mathbf Q}
\newcommand{\bR}{\mathbf R}

% Blackboard (hollow) math

\newcommand{\hE}{\mathbb E}
\newcommand{\hP}{\mathbb P}

% Greek and misc symbols

\newcommand{\0}{\varnothing}

\renewcommand{\a}{\alpha}
\renewcommand{\b}{\beta}
\renewcommand{\d}{\delta}
\newcommand{\e}{\varepsilon}
\newcommand{\f}{\varphi}
\newcommand{\g}{\gamma}
\renewcommand{\l}{\lambda}
\newcommand{\m}{\mu}
\newcommand{\s}{\sigma}
\renewcommand{\th}{\theta}

\newcommand{\D}{\Delta}

% Delimiters

%\newcommand{\defeq}{\coloneqq}
\newcommand*{\defeq}{\mathrel{\vcenter{\baselineskip0.5ex \lineskiplimit0pt
                     \hbox{\scriptsize.}\hbox{\scriptsize.}}}
                     =}
\newcommand{\<}{\langle}
\renewcommand{\>}{\rangle}

\title{Analysis II}
\author{Trong}
\date{December 1, 2018---\today}

\begin{document}
\pdfrender{StrokeColor=black,TextRenderingMode=2,LineWidth=0.1pt}
\sloppy
\maketitle

% \usepackage{hyperref} for hyper links in table of contents. Remember to
% compile twice to update table of contents
\tableofcontents

\section{Differentiation}

\begin{definition}[Directional derivative]
  Let \( A \subset \bR^m, f: A\longrightarrow \bR^n. \) Suppose $A$ contains a neighbourhood of \( a. \) Given \( u \in \bR^m \) with \( u \neq 0, \) define the directional derivative of \( f \) at \( a \) in the direction of \( u \) to be
  \[
  f'(a; u) = \lim_{t\to 0} \frac{f(a+u) - f(a)}{t}
  \]
if it exists.
\end{definition}

\begin{definition}[Derivative]
  Let \( A \subset \bR^m, f: A\longrightarrow \bR^n. \) Suppose $A$ contains a
  neighbourhood of a. We say that \( f \) is differentiable at \( a \) if there
  is an \( n\times m \) matrix $B$ s.t.
\[
\frac{f(a+h) - f(a) - B\cdot h}{|h|} \longrightarrow 0
\]
as \( h \longrightarrow 0. \) The matrix $B$ is unique and is denoted \( D f(a).
\) Sometimes people also call the derivative the gradient, and write \( D f(a) =
\nabla f(a). \)
\end{definition}

\begin{theorem}[Relating directional derivatives to the derivative of \( f \)]
  Let \( A \subset \bR^m, f: A\longrightarrow \bR^n. \) If f is differentiable at \( a, \) then all the directional derivatives of \( f \) at \( a \) exists, and
\[
f'(a; u) = D f(a) \cdot u.
\]
\end{theorem}

\begin{definition}[Partial derivative]
  Let \( A \subset \bR^m, f: A \longrightarrow \bR. \) Define the \( j \)-th partial derivative of \( f \) at \( a \) to be directional derivative of \( f \) at \( a \) with respect to the vector \( e_j, \) provided it exists; and we denote it by \( D_j f(a): \)
  \[
  D_j f(a) = \lim_{t\to 0} \frac{f(a + t e_j) - f(a)}{t}.
  \]
\end{definition}

IOW, partial derivatives are directional derivatives along coordinate axes. Note
that if we define \( \phi(t) = f(a_1,\ldots, a_{j-1}, t, a_{j+1},\ldots,a_m), \)
then
\[
D_j f(a) = \phi'(a_j).
\]

\begin{theorem}[Derivative of a real-valued function]
  Let \( A \subset \bR^m, f: A \longrightarrow \bR. \) If \( f \) is differentiable at \( a, \) then the derivative of \( f \) is the row matrix
\[
Df(a) = [D_1 f(a) \quad \cdots \quad D_m f(a)].
\]
\end{theorem}

\begin{theorem}
Let \( A \subset \bR^m, f: A \longrightarrow \bR^n. \) Suppose \( A \) contains a neighbourhood of \( a. \) Let \( f_i: A \longrightarrow R \) be the \( i \)-th component function of \( f, \) so that
\[
f(x) = \begin{bmatrix}
f_1 (x) \\
\vdots \\
f_n (x)
\end{bmatrix}.
\]
\begin{itemize}
\item Then \( f \) is differentiable at \( a \) iff each component \( f_i \) is differentiable at \( a. \)

\item If \( f \) is differentiable at \( a, \) then its derivative is the \( n \times m \) matrix whose \( i \)-th row is the derivative of \( f_i,  \) i.e.

\footnotesize
\[
D f(a) = \begin{bmatrix}
D f_1 (a) \\
\vdots \\
D f_n (a)
\end{bmatrix} = \begin{bmatrix}
D_1 f_1 (a) & \cdots & D_m f_1 (a) \\
\vdots & & \vdots \\
D_1 f_n (a) & \cdots & D_m f_n (a)
\end{bmatrix}.
\]
\normalsize
  This matrix is called the Jacobian matrix of \( f. \)
\end{itemize}
\end{theorem}

Roughly: Differentiability of \( f: \bR^m \longrightarrow \bR^n \) is equivalent
to differentiability of each component, because the components are independent
of each other as far as taking limits is concerned. Note that this doesn't imply
that the partial derivatives of the components must be continuous, only that
they exist.

\section{Continuously differentiable functions}

\begin{theorem}[Mean value theorem]
If $ \phi: [a, b] \longrightarrow \bR $ is continuous at each point of the closed interval $ [a, b], $ and differentiable at each point of the interval $ (a, b), $ then there exists a point $ c $ of $ (a, b) $ s.t. $$
  \phi(b) - \phi(a) = \phi'(c) (b - a).
$$
\end{theorem}

\begin{theorem}[Continuously differentiable functions]
  Let $ A $ be open in $ \bR^m. $ Suppose that the partial derivatives $ D_j f_i(x) $ of the component functions of $ f $ exist at each point $ x \in A $ and are continuous on $ A. $ Then $ f $ is differentiable at each point of $ A. $
\end{theorem}

This theorem guarantees differentiability of $ f $ if its partial derivatives exist and are continuous. Such a function is called continuously differentiable, or $ C^1 $ on $ A. $

\begin{theorem}
  Let $ A $ be open in $ \bR^m, f: A \longrightarrow \bR $ be a function of class $ C^2. $ Then for each $ a \in A, $ the mixed second order partial derivatives are equal: $$
    D_k D_j f(a) = D_j D_k f(a).
  $$
\end{theorem}

\break

\section{Homogeneous functions and the Chain Rule}

\begin{proposition}
A function $ f: \bR^n \longrightarrow \bR $ is called homogeneous of degree $ m $ if $ f(tx) = t^m f(x) $ for all $ x $ and $ t. $ If $ f $ is also differentiable, then \[
\sum_{i=1}^n x_i D_i f(x) = m f(x). \tag{1}
\]
\end{proposition}

\begin{proof}
Let $ g(t) = f(tx) = t^m f(x). $ By the Chain Rule, \begin{align*}
g'(t) &= \sum_{i} D_i f(tx) \frac{d}{dt} (tx_i) \\
&= \sum_{i} x_i D_i f(tx) \\
&= m t^{m-1} f(x).
\end{align*}
Plugging in $ t = 1 $ yields the desired result.
\end{proof}

\begin{example}
Intuitively a homogeneous function transforms scaling from its arguments to its value. Examples are linear operators: $ T(ax) = aT(x), $ which are homogeneous of order 1.

Another example is the function \[
  f(x_1, x_2) = x_1^2 + x_2^2,
\]
which is homogeneous of order 2, since \[
f(t x_1, t x_2) = t^2 x_1^2 + t^2 x_2^2 = t^2 f(x_1, x_2).
\]
Let's verify (1): \begin{align*}
\sum_{i=1}^n x_i D_i f(x) &= 2x_1 x_1 + 2x_2 x_2 = 2 f(x).
\end{align*}
\end{example}

\begin{proposition}
If $ f: \bR^n \longrightarrow \bR $ is differentiable and $ f(0) = 0, $ then there exist $ g_i: \bR^n \longrightarrow \bR $ s.t. \[
f(x) = \sum_{i=1}^n x_i g_i (x). \tag{2}
\]
Specifically, \[
f(x) = \sum_{i=1}^n x_i \int_0^1 D_i f(tx) dt. \tag{3}
\]
\end{proposition}

\begin{proof}
Let $ h(t) = f(tx). $ Then \begin{align*}
\int_0^1 h'(t) dt = h(1) - h(0) = f(x) - f(0) = f(x).
\end{align*}
On the other hand, by the Chain Rule, \[
h'(t) = \sum_{i=1}^n x_i D_i f(tx),
\]
hence \begin{align*}
\int_0^1 h'(t) dt &= \int_0^1 \sum_{i=1}^n x_i D_i f(tx) dt \\
&= \sum_{i=1}^n x_i \int_0^1 D_i f(tx) dt.
\end{align*}
Thus \[
g_i (x) = \int_0^1 D_i f(tx) dt
\] are the functions we're looking for.
\end{proof}

\begin{example}
It's always good to verify an abstract result on a simple example: \[
f(x_1, x_2) = x_1^2 + x_2^2.
\]
Formula (2) says \[
f(x) = x_1 g_1 (x) + x_2 g_2 (x),
\]
which would suggest that \[
g_1(x) = x_1 \quad \text{and} \quad g_2(x) = x_2.
\]
Let's check: \[
D_1 f(tx) = 2 x_1 |_{tx} = 2 t x_1,
\]
which we plug into $ g_1 $: \begin{align*}
g_1 (x) &= \int_0^1 D_1 f(tx) dt \\
&= \int_0^1 2 t x_1 dt \\
&= t^2 x_1 |_0^1 \\
&= x_1.
\end{align*}
Similarly we find that $ g_2 (x) = x_2, $ as required.
\end{example}

\begin{note}
If $ f: \bR^n \longrightarrow \bR $ is differentiable and homogeneous of degree $ m, $ then $ f(0) = 0 $ since $ f(0) = f(t \cdot 0) = t^m f(0). $ Therefore (3) holds: \[
  f(x) = \sum_{i=1}^n x_i \int_0^1 D_i f(tx) dt.
\]
Combining it with (1)--- \[
\sum_{i=1}^n x_i D_i f(x) = m f(x),
\] we're led to believe that \[
\frac{D_i f(x)}{m} = \int_0^1 D_i f(tx) dt.
\]
This we can easily check for $ f(x_1, x_2) = x_1^2 + x_2^2 $ by reusing some of our previous calculations, e.g. \[
  \frac{D_1 f(x)}{m} = \frac{2 x_1}{2} = x_1 = \int_0^1 D_1 f(tx) dt.
\]
\end{note}

\end{document}
