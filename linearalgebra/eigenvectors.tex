\documentclass[11pt]{article}
\usepackage{geometry}                % See geometry.pdf to learn the layout options. There are lots.
\geometry{letterpaper}                   % ... or a4paper or a5paper or ... 
%\geometry{landscape}                % Activate for for rotated page geometry
%\usepackage[parfill]{parskip}    % Activate to begin paragraphs with an empty line rather than an indent
\usepackage{graphicx}
\usepackage{amsmath,amssymb,amsfonts,amsthm} 
\usepackage{epstopdf}

% Computer Concrete
%\usepackage{concmath}
%\usepackage[T1]{fontenc}

% Times variants
%
%\usepackage{mathptmx}
%\usepackage[T1]{fontenc}
%
%\usepackage[T1]{fontenc}
%\usepackage{stix}
%
% Needs to typeset using LuaLaTeX:
%\usepackage{unicode-math}
%\setmainfont{XITS}
%\setmathfont{XITS Math}

\DeclareGraphicsRule{.tif}{png}{.png}{`convert #1 `dirname #1`/`basename #1 .tif`.png}

\theoremstyle{plain}
\newtheorem{thm}{Theorem}
\newtheorem{cor}[thm]{Corollary}
\newtheorem{lem}[thm]{Lemma}
\newtheorem{prop}[thm]{Proposition}
\newtheorem{conj}[thm]{Conjecture}

\theoremstyle{definition}
\newtheorem{defn}[thm]{Definition}
\newtheorem{e.g.}[thm]{Example}
\newtheorem*{keywords}{Keywords}

\theoremstyle{remark}
\newtheorem{rem}[thm]{Remark}
\newtheorem{note}[thm]{Note}

\title{Eigenvectors and Eigenvalues}
\author{N. Trong}
\date{\today}                                           % Activate to display a given date or no date

\begin{document}
\maketitle
%\section{}
%\subsection{}

\begin{prop}
Eigenvectors corresponding to distinct eigenvalues are linearly independent.
\end{prop}

\begin{proof}
Easy to show for two eigenvectors, then use induction.
\end{proof}

\begin{conj}
A matrix $A$ is diagonalizable iff the dimensions of its eigenspaces---i.e. the geometric multiplicities over all its eigenvalues---add up to the size of $A$. In this case the geometric multiplicity of each eigenvalue is equal to its algebraic multiplicity.
\end{conj}

\begin{prop}
Let $T$ be a linear operator on a finite-dimensional vector space $V$, and let $\beta$ be an ordered basis for $V$. Then $\lambda$ is an eigenvalue of $T$ iff it is an eigenvalue of $[T]_\beta$.
\end{prop}

\begin{cor}
Similar matrices have the same eigenvalues, but not necessarily the same eigenvectors.
\end{cor}

\begin{prop}
If $v$ is an eigenvector of $A$ corresponding to eigenvalue $\lambda$, and $B$ is similar to $A$ under change of coordinates matrix $Q$, then $Qv$ is an eigenvector of $B$ corresponding to the same eigenvalue $\lambda$.  Another way of saying this is that change of coordinates preserves eigenvalues and eigenvectors.
\end{prop}

\begin{proof}
Let $A = Q^{-1}BQ$. Then
\begin{align*}
Av &= Q^{-1}BQv \\
QAv &= BQv \\
\lambda Qv &= BQv,
\end{align*}
so $Qv$ is an eigenvector corresponding to $\lambda$ of $B$.
\end{proof}

\begin{keywords}
Eigenvectors, eigenvalues, differential operator, eigenfunctions, algebraic and geometric multiplicities, change of coordinates matrix.
\end{keywords}

\end{document}  