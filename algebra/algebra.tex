% Converting a PDF into JPGs:
%
% gs -dFirstPage=26 -dLastPage=33 -dBATCH -dNOPAUSE -sDEVICE=jpeg -r666 -dUseCropBox -o images/OUT-%04d.jpg IN.pdf

% Alternatively, output at a lower resolution, and use convert to blur, for a
% more book-like effect:
%
% gs -dFirstPage=1 -dLastPage=10 -dBATCH -dNOPAUSE -sDEVICE=jpeg -r150 -dUseCropBox -o images/RLII-%04d.jpg ReinforcementLearningII.pdf
% mogrify -blur 0x0.5 images/RLII-*.jpg

\documentclass[12pt]{article}
% \documentclass[14pt]{extarticle}
% \documentclass[17pt]{extarticle}
% \documentclass[20pt]{extarticle}
% \usepackage[margin=0.5cm,paperwidth=4in,paperheight=4in]{geometry}
% \usepackage[margin=1cm,paperwidth=7in,paperheight=7in]{geometry}
% \usepackage[margin=1cm,paperwidth=8in,paperheight=10in]{geometry}
% \usepackage[a4paper,margin=1cm,paperwidth=8.5in,paperheight=11in]{geometry}

% Font sizes for reference
% \tiny
% \scriptsize
% \footnotesize
% \small
% \normalsize
% \large
% \Large
% \LARGE
% \huge
% \Huge

% Languages: English, Vietnamese, Chinese, Japanese, Korean.

% \usepackage[english,vietnamese]{babel}
% \usepackage[utf8]{inputenc}
% \usepackage[T1,T2A]{fontenc}
% \usepackage{CJKutf8}
% % Options for fonts: Chinese: gbsn, gkai; Japanese: min, goth, maru
% \begin{CJK}{UTF8}{gbsn}天下人都\end{CJK}

\usepackage{float}
\usepackage{graphicx}
\graphicspath{{images/}}

\usepackage{listings}
\lstset{basicstyle=\ttfamily, aboveskip=\bigskipamount, belowskip=\bigskipamount}

\usepackage{amsmath,amssymb,amsfonts,amsthm}
\usepackage{epstopdf}
\usepackage{epigraph}
\usepackage{url}
\usepackage{mathtools}

\usepackage{hyperref}
\hypersetup{
    colorlinks,
    citecolor=red,
    filecolor=red,
    linkcolor=red,
    urlcolor=red
}

\usepackage{pdfrender,xcolor}

\usepackage{lmodern}
\usepackage[T1]{fontenc}

% Computer Concrete
%\usepackage{concmath}
%\usepackage[T1]{fontenc}

% Times variants

% \usepackage{mathptmx}
% \usepackage[T1]{fontenc}

% \usepackage[T1]{fontenc}
% \usepackage{stix}

% Needs to typeset using LuaLaTeX:

% \usepackage{unicode-math}
% \setmainfont{XITS}
% \setmathfont{XITS Math}

% garamond

% \usepackage[cmintegrals,cmbraces]{newtxmath}
% \usepackage{ebgaramond-maths}
% \usepackage[T1]{fontenc}

\DeclareGraphicsRule{.tif}{png}{.png}{`convert #1 `dirname #1`/`basename #1 .tif`.png}

% Environments

% Redefine enumerate environment to have less spacing between items
\let\oldBeginEnumerate=\enumerate
\let\oldEndEnumerate=\endenumerate
\renewenvironment{enumerate}{
\oldBeginEnumerate
  \setlength{\itemsep}{0pt}
  \setlength{\parskip}{0pt}
  \setlength{\parsep}{0pt}
}{\oldEndEnumerate}

% Italics body
\theoremstyle{plain}
\newtheorem{theorem}{Theorem}
\newtheorem{corollary}[theorem]{Corollary}
\newtheorem{lemma}[theorem]{Lemma}
\newtheorem{proposition}[theorem]{Proposition}
\newtheorem{conjecture}[theorem]{Conjecture}
\newtheorem{question}[theorem]{Question}
\newtheorem{problem}[theorem]{Problem}
\newtheorem{exercise}[theorem]{Exercise}
\newtheorem{definition}[theorem]{Definition}

% Regular body
\theoremstyle{definition}
\newtheorem{example}[theorem]{Example}
\newtheorem{keywords}{Keywords}
\newtheorem{reference}{Reference}

\theoremstyle{remark}
\newtheorem{remark}[theorem]{Remark}
\newtheorem{note}[theorem]{Note}

\newenvironment{solution}{\begin{proof}[Solution]}{\end{proof}}

% Black square for QED symbol
\renewcommand{\qedsymbol}{$\blacksquare$}

% Thicker fraction divider
\renewcommand\frac[2]{\genfrac{}{}{0.6pt}{}{#1}{#2}}

% Operators

\DeclareMathOperator{\argmax}{argmax}

% Bold math

\newcommand{\bE}{\mathbf E}
\newcommand{\bN}{\mathbf N}
\newcommand{\bQ}{\mathbf Q}
\newcommand{\bR}{\mathbf R}

% Blackboard (hollow) math

\newcommand{\hE}{\mathbb E}
\newcommand{\hP}{\mathbb P}

% Greek and misc symbols

\newcommand{\0}{\varnothing}

\renewcommand{\a}{\alpha}
\renewcommand{\b}{\beta}
\renewcommand{\d}{\delta}
\newcommand{\e}{\varepsilon}
\newcommand{\f}{\varphi}
\newcommand{\g}{\gamma}
\renewcommand{\l}{\lambda}
\newcommand{\m}{\mu}
\newcommand{\s}{\sigma}
\renewcommand{\th}{\theta}

\newcommand{\D}{\Delta}

% Delimiters

%\newcommand{\defeq}{\coloneqq}
\newcommand*{\defeq}{\mathrel{\vcenter{\baselineskip0.5ex \lineskiplimit0pt
                     \hbox{\scriptsize.}\hbox{\scriptsize.}}}
                     =}
\newcommand{\<}{\langle}
\renewcommand{\>}{\rangle}

\title{Abstract Algebra}
\author{Trong}
\date{\today}

\begin{document}
\pdfrender{StrokeColor=black,TextRenderingMode=2,LineWidth=0.1pt}
\sloppy
\maketitle
% \thispagestyle{empty}
% \pagestyle{empty}

% \usepackage{hyperref} for hyper links in table of contents. Remember to
% compile twice to update table of contents
\tableofcontents

\section{M\"obius Function}

\begin{proposition}
For every natural number $ n $ define the M\"obius function
\[
\m(n) = \left\{
\begin{array}{llr}
1 & \text{ if $ n = 1 $} & (I) \\
0 & \text{ if $ p^2 | n $ for some prime $ p $} & (II) \\
(-1)^k & \text{ if $ n = p_1 \cdots p_k $ for distinct primes $ p_i. $} & (III)
\end{array}
\right.
\]
Then $ \m $ is multiplicative, i.e. for $ (m, n) = 1, $ \[
\m(mn) = \m(m) \m(n).
\]
Furthermore, \[
\f(n) = \sum_{d|n} \m(d) \frac{n}{d}. \tag{IV}
\]
\end{proposition}

\begin{proof}
Multiplicativity is easy to check if either $ m $ or $ n $ satisfies (I) or (II). Therefore suppose $ m = p_1 \cdots p_k $ and $ n = q_1 \cdots q_l $ are each a product of distinct primes. Since $ (m, n) = 1, $ $ p_1, \ldots, p_k, q_1, \ldots, q_l $ are in fact all distinct primes. Then \begin{align*}
\m(mn) = (-1)^{k+l} = (-1)^{k} (-1)^{l} = \m(m)\m(n).
\end{align*}

To show (IV), recall the formula
\begin{align*}
\f(n) &= n \Big(1 - \frac{1}{p_1}\Big) \cdots \Big(1 - \frac{1}{p_k}\Big) \\
&= n - \frac{n}{p_1} - \cdots + \frac{n}{p_1 p_2} + \cdots + (-1)^{k} \frac{n}{p_1\cdots p_k}, \tag{V}
\end{align*}
where $ p_i $ are all the distinct primes of $ n. $ Note that every term in this expansion has the form \[
  (-1)^{j} \frac{n}{q_1 \cdots q_j} = \m(d) \frac{n}{d}
\]
where the $ q_i $ are some subset of $ p_1,\ldots,p_k. $ This accounts for the divisors $ d $ of $ n $ of the form (I) and (III). We can ignore divisors of the form (II) in (IV) since in those cases $ \m(d) = 0. $ Therefore the terms in (V) are precisely the same ones in (IV).
\end{proof}

\section{Semigroup}

\begin{definition}
A semigroup is a set $ S $ with a product which associates to each ordered pair $ a, b\in S $ a product $ ab $ s.t. associativity holds: $ (ab)c = a(bc) $ for any $ a,b,c \in S. $ In other words, a semigroup is like a group without existence of an identity or inverses.
\end{definition}

\begin{example}
The set of all mappings of a set $ X $ to itself forms a semigroup in which the product is composition of mappings. The set of all one-to-one mappings of a set $ X $ to itself forms a group under composition.
\end{example}

\begin{proof}
Composition of mappings is associative. One-to-one mappings furnish the identity map and inverses.
\end{proof}

\begin{proposition}
Suppose $ S $ is a semigroup with a finite number of elements that obey the Cancellation Laws: if either $ ab = ac $ or $ ba = ca, $ then $ b = c. $ Then $ S $ is a group.
\end{proposition}

\begin{proof}
For simplicity let $ S = \{ a, b, c, d, e \} $ consist of 5 elements, where 5 is arbitrary. First we want to show that $ S $ contains an identity element. Consider the elements \[
a, aa, aaa, 4a, 5a, 6a.
\]
Since $ S $ is finite, by Pigeonhole two of these must be the same, say \[
2a = 5a.
\]
By Cancellation, \[
a = 4a,
\]
so $ 3a $ is our tentative identity, at least as far as $ a $ is concerned. Now to show that it works for $ b $ as well: \begin{align*}
2a &= 5a \\
2a b &= 5a b \\
b &= 3a b,
\end{align*}
IOW, $ 3a $ is an identity for $ b $ too. Arguing similarly for the other elements, we see that $ 3a $ is an identity for $ S. $ Similar arguments will show that every element of $ S $ has an inverse, and hence $ S $ is a group.
\end{proof}

\end{document}
