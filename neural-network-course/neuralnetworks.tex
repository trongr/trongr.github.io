\documentclass[12pt]{article}
\usepackage{geometry}                % See geometry.pdf to learn the layout options. There are lots.
\geometry{letterpaper}                   % ... or a4paper or a5paper or ... 
%\geometry{landscape}                % Activate for for rotated page geometry
%\usepackage[parfill]{parskip}    % Activate to begin paragraphs with an empty line rather than an indent
\usepackage{float}
\usepackage{graphicx}
\graphicspath{{images/}}
\usepackage{amsmath,amssymb,amsfonts,amsthm}
\usepackage{epstopdf}
\usepackage{cleveref}
\usepackage{epigraph}
\usepackage{url}

% Computer Concrete
%\usepackage{concmath}
%\usepackage[T1]{fontenc}

% Times variants
%
\usepackage{mathptmx}
\usepackage[T1]{fontenc}
%
%\usepackage[T1]{fontenc}
%\usepackage{stix}
%
% Needs to typeset using LuaLaTeX:
%\usepackage{unicode-math}
%\setmainfont{XITS}
%\setmathfont{XITS Math}

% garamond
%\usepackage[cmintegrals,cmbraces]{newtxmath}
%\usepackage{ebgaramond-maths}
%\usepackage[T1]{fontenc}

\DeclareGraphicsRule{.tif}{png}{.png}{`convert #1 `dirname #1`/`basename #1 .tif`.png}

\theoremstyle{plain}
\newtheorem{theorem}{Theorem}
\newtheorem{corollary}[theorem]{Corollary}
\newtheorem{lemma}[theorem]{Lemma}
\newtheorem{proposition}[theorem]{Proposition}
\newtheorem{conjecture}[theorem]{Conjecture}
\newtheorem{question}[theorem]{Question}

\theoremstyle{definition}
\newtheorem{definition}[theorem]{Definition}
\newtheorem{example}[theorem]{Example}
\newtheorem{keywords}{Keywords}
\newtheorem{reference}{Reference}
\newtheorem{todo}{TODO}

\theoremstyle{remark}
\newtheorem{remark}[theorem]{Remark}
\newtheorem{note}[theorem]{Note}

\title{Neural Networks Notes}
\author{Nhan Trong}
\date{\today}                                           % Activate to display a given date or no date

\begin{document}
\maketitle

%\epigraph{\textit{}}{}

\part{Different Types of Neurons and Learning}

\begin{keywords}
Fruit flies, MNIST, TIMIT, linear, binary threshold, rectified / linear threshold, logistic, stochastic binary neurons, supervised, unsupervised, reinforcement learning.
\end{keywords}

\begin{question}
How many other neurons does a neuron talk to? Do they change neighbours?
\end{question}

\begin{note}
``Goal of unsupervised learning: provides a compact, low-dimensional representation of the input,'' like Pied Piper's compression algorithm using neural networks!
\end{note}

\part{Neural Network Architectures}

\begin{keywords}
Feed forward, recurrent, symmetrically connected neural network, perceptrons, convexity condition.
\end{keywords}

\begin{reference}
\url{http://www.cs.toronto.edu/~rgrosse/csc321/homework.html}.
\end{reference}

\end{document}
